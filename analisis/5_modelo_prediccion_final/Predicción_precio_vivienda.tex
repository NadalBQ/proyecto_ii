% Options for packages loaded elsewhere
\PassOptionsToPackage{unicode}{hyperref}
\PassOptionsToPackage{hyphens}{url}
%
\documentclass[
]{article}
\usepackage{amsmath,amssymb}
\usepackage{iftex}
\ifPDFTeX
  \usepackage[T1]{fontenc}
  \usepackage[utf8]{inputenc}
  \usepackage{textcomp} % provide euro and other symbols
\else % if luatex or xetex
  \usepackage{unicode-math} % this also loads fontspec
  \defaultfontfeatures{Scale=MatchLowercase}
  \defaultfontfeatures[\rmfamily]{Ligatures=TeX,Scale=1}
\fi
\usepackage{lmodern}
\ifPDFTeX\else
  % xetex/luatex font selection
\fi
% Use upquote if available, for straight quotes in verbatim environments
\IfFileExists{upquote.sty}{\usepackage{upquote}}{}
\IfFileExists{microtype.sty}{% use microtype if available
  \usepackage[]{microtype}
  \UseMicrotypeSet[protrusion]{basicmath} % disable protrusion for tt fonts
}{}
\makeatletter
\@ifundefined{KOMAClassName}{% if non-KOMA class
  \IfFileExists{parskip.sty}{%
    \usepackage{parskip}
  }{% else
    \setlength{\parindent}{0pt}
    \setlength{\parskip}{6pt plus 2pt minus 1pt}}
}{% if KOMA class
  \KOMAoptions{parskip=half}}
\makeatother
\usepackage{xcolor}
\usepackage[margin=1in]{geometry}
\usepackage{color}
\usepackage{fancyvrb}
\newcommand{\VerbBar}{|}
\newcommand{\VERB}{\Verb[commandchars=\\\{\}]}
\DefineVerbatimEnvironment{Highlighting}{Verbatim}{commandchars=\\\{\}}
% Add ',fontsize=\small' for more characters per line
\usepackage{framed}
\definecolor{shadecolor}{RGB}{248,248,248}
\newenvironment{Shaded}{\begin{snugshade}}{\end{snugshade}}
\newcommand{\AlertTok}[1]{\textcolor[rgb]{0.94,0.16,0.16}{#1}}
\newcommand{\AnnotationTok}[1]{\textcolor[rgb]{0.56,0.35,0.01}{\textbf{\textit{#1}}}}
\newcommand{\AttributeTok}[1]{\textcolor[rgb]{0.13,0.29,0.53}{#1}}
\newcommand{\BaseNTok}[1]{\textcolor[rgb]{0.00,0.00,0.81}{#1}}
\newcommand{\BuiltInTok}[1]{#1}
\newcommand{\CharTok}[1]{\textcolor[rgb]{0.31,0.60,0.02}{#1}}
\newcommand{\CommentTok}[1]{\textcolor[rgb]{0.56,0.35,0.01}{\textit{#1}}}
\newcommand{\CommentVarTok}[1]{\textcolor[rgb]{0.56,0.35,0.01}{\textbf{\textit{#1}}}}
\newcommand{\ConstantTok}[1]{\textcolor[rgb]{0.56,0.35,0.01}{#1}}
\newcommand{\ControlFlowTok}[1]{\textcolor[rgb]{0.13,0.29,0.53}{\textbf{#1}}}
\newcommand{\DataTypeTok}[1]{\textcolor[rgb]{0.13,0.29,0.53}{#1}}
\newcommand{\DecValTok}[1]{\textcolor[rgb]{0.00,0.00,0.81}{#1}}
\newcommand{\DocumentationTok}[1]{\textcolor[rgb]{0.56,0.35,0.01}{\textbf{\textit{#1}}}}
\newcommand{\ErrorTok}[1]{\textcolor[rgb]{0.64,0.00,0.00}{\textbf{#1}}}
\newcommand{\ExtensionTok}[1]{#1}
\newcommand{\FloatTok}[1]{\textcolor[rgb]{0.00,0.00,0.81}{#1}}
\newcommand{\FunctionTok}[1]{\textcolor[rgb]{0.13,0.29,0.53}{\textbf{#1}}}
\newcommand{\ImportTok}[1]{#1}
\newcommand{\InformationTok}[1]{\textcolor[rgb]{0.56,0.35,0.01}{\textbf{\textit{#1}}}}
\newcommand{\KeywordTok}[1]{\textcolor[rgb]{0.13,0.29,0.53}{\textbf{#1}}}
\newcommand{\NormalTok}[1]{#1}
\newcommand{\OperatorTok}[1]{\textcolor[rgb]{0.81,0.36,0.00}{\textbf{#1}}}
\newcommand{\OtherTok}[1]{\textcolor[rgb]{0.56,0.35,0.01}{#1}}
\newcommand{\PreprocessorTok}[1]{\textcolor[rgb]{0.56,0.35,0.01}{\textit{#1}}}
\newcommand{\RegionMarkerTok}[1]{#1}
\newcommand{\SpecialCharTok}[1]{\textcolor[rgb]{0.81,0.36,0.00}{\textbf{#1}}}
\newcommand{\SpecialStringTok}[1]{\textcolor[rgb]{0.31,0.60,0.02}{#1}}
\newcommand{\StringTok}[1]{\textcolor[rgb]{0.31,0.60,0.02}{#1}}
\newcommand{\VariableTok}[1]{\textcolor[rgb]{0.00,0.00,0.00}{#1}}
\newcommand{\VerbatimStringTok}[1]{\textcolor[rgb]{0.31,0.60,0.02}{#1}}
\newcommand{\WarningTok}[1]{\textcolor[rgb]{0.56,0.35,0.01}{\textbf{\textit{#1}}}}
\usepackage{graphicx}
\makeatletter
\def\maxwidth{\ifdim\Gin@nat@width>\linewidth\linewidth\else\Gin@nat@width\fi}
\def\maxheight{\ifdim\Gin@nat@height>\textheight\textheight\else\Gin@nat@height\fi}
\makeatother
% Scale images if necessary, so that they will not overflow the page
% margins by default, and it is still possible to overwrite the defaults
% using explicit options in \includegraphics[width, height, ...]{}
\setkeys{Gin}{width=\maxwidth,height=\maxheight,keepaspectratio}
% Set default figure placement to htbp
\makeatletter
\def\fps@figure{htbp}
\makeatother
\setlength{\emergencystretch}{3em} % prevent overfull lines
\providecommand{\tightlist}{%
  \setlength{\itemsep}{0pt}\setlength{\parskip}{0pt}}
\setcounter{secnumdepth}{-\maxdimen} % remove section numbering
\usepackage{booktabs}
\usepackage{longtable}
\usepackage{array}
\usepackage{multirow}
\usepackage{wrapfig}
\usepackage{float}
\usepackage{colortbl}
\usepackage{pdflscape}
\usepackage{tabu}
\usepackage{threeparttable}
\usepackage{threeparttablex}
\usepackage[normalem]{ulem}
\usepackage{makecell}
\usepackage{xcolor}
\ifLuaTeX
  \usepackage{selnolig}  % disable illegal ligatures
\fi
\usepackage{bookmark}
\IfFileExists{xurl.sty}{\usepackage{xurl}}{} % add URL line breaks if available
\urlstyle{same}
\hypersetup{
  pdftitle={Predicción},
  hidelinks,
  pdfcreator={LaTeX via pandoc}}

\title{Predicción}
\author{}
\date{\vspace{-2.5em}2025-05-22}

\begin{document}
\maketitle

\begin{Shaded}
\begin{Highlighting}[]
\FunctionTok{library}\NormalTok{(dplyr)}
\end{Highlighting}
\end{Shaded}

\begin{verbatim}
## 
## Adjuntando el paquete: 'dplyr'
\end{verbatim}

\begin{verbatim}
## The following objects are masked from 'package:stats':
## 
##     filter, lag
\end{verbatim}

\begin{verbatim}
## The following objects are masked from 'package:base':
## 
##     intersect, setdiff, setequal, union
\end{verbatim}

\begin{Shaded}
\begin{Highlighting}[]
\FunctionTok{library}\NormalTok{(readr)}
\FunctionTok{library}\NormalTok{(randomForest)}
\end{Highlighting}
\end{Shaded}

\begin{verbatim}
## randomForest 4.7-1.2
\end{verbatim}

\begin{verbatim}
## Type rfNews() to see new features/changes/bug fixes.
\end{verbatim}

\begin{verbatim}
## 
## Adjuntando el paquete: 'randomForest'
\end{verbatim}

\begin{verbatim}
## The following object is masked from 'package:dplyr':
## 
##     combine
\end{verbatim}

\begin{Shaded}
\begin{Highlighting}[]
\FunctionTok{library}\NormalTok{(Metrics)}
\end{Highlighting}
\end{Shaded}

\begin{verbatim}
## Warning: package 'Metrics' was built under R version 4.4.3
\end{verbatim}

\begin{Shaded}
\begin{Highlighting}[]
\FunctionTok{library}\NormalTok{(ggplot2)}
\end{Highlighting}
\end{Shaded}

\begin{verbatim}
## 
## Adjuntando el paquete: 'ggplot2'
\end{verbatim}

\begin{verbatim}
## The following object is masked from 'package:randomForest':
## 
##     margin
\end{verbatim}

\begin{Shaded}
\begin{Highlighting}[]
\FunctionTok{library}\NormalTok{(tidyr)}
\FunctionTok{library}\NormalTok{(tibble)}
\FunctionTok{library}\NormalTok{(stringr)}
\FunctionTok{library}\NormalTok{(broom)}
\FunctionTok{library}\NormalTok{(knitr)}
\FunctionTok{library}\NormalTok{(kableExtra)}
\end{Highlighting}
\end{Shaded}

\begin{verbatim}
## Warning: package 'kableExtra' was built under R version 4.4.3
\end{verbatim}

\begin{verbatim}
## 
## Adjuntando el paquete: 'kableExtra'
\end{verbatim}

\begin{verbatim}
## The following object is masked from 'package:dplyr':
## 
##     group_rows
\end{verbatim}

\begin{Shaded}
\begin{Highlighting}[]
\FunctionTok{library}\NormalTok{(xgboost)}
\end{Highlighting}
\end{Shaded}

\begin{verbatim}
## Warning: package 'xgboost' was built under R version 4.4.3
\end{verbatim}

\begin{verbatim}
## 
## Adjuntando el paquete: 'xgboost'
\end{verbatim}

\begin{verbatim}
## The following object is masked from 'package:dplyr':
## 
##     slice
\end{verbatim}

\begin{Shaded}
\begin{Highlighting}[]
\FunctionTok{library}\NormalTok{(purrr)}
\end{Highlighting}
\end{Shaded}

\begin{Shaded}
\begin{Highlighting}[]
\NormalTok{barrios\_reseñas }\OtherTok{\textless{}{-}} \FunctionTok{read\_csv}\NormalTok{(}\StringTok{\textquotesingle{}barriosResenas.csv\textquotesingle{}}\NormalTok{, }\AttributeTok{show\_col\_types =} \ConstantTok{FALSE}\NormalTok{) }\SpecialCharTok{\%\textgreater{}\%}  
  \FunctionTok{rename}\NormalTok{(}\AttributeTok{barrio =}\NormalTok{ neighbourhood\_cleansed) }\SpecialCharTok{\%\textgreater{}\%}
  \FunctionTok{rename}\NormalTok{(puntuacion\_media\_normal\_reseñas }\OtherTok{=}\NormalTok{ puntuacion\_media\_normal) }\SpecialCharTok{\%\textgreater{}\%}
  \FunctionTok{rename}\NormalTok{(puntuacion\_media\_reseñas }\OtherTok{=}\NormalTok{ puntuacion\_media)}
  
\CommentTok{\# Cargar datos fiabilidad y pisos}
\NormalTok{pisos\_fiabilidad }\OtherTok{\textless{}{-}} \FunctionTok{read\_csv}\NormalTok{(}\StringTok{\textquotesingle{}pisos\_con\_analisis\_fiabilidad.csv\textquotesingle{}}\NormalTok{, }\AttributeTok{show\_col\_types =} \ConstantTok{FALSE}\NormalTok{) }\SpecialCharTok{\%\textgreater{}\%}
  \FunctionTok{rename}\NormalTok{(}\AttributeTok{barrio =}\NormalTok{ neighbourhood\_cleansed) }\SpecialCharTok{\%\textgreater{}\%}
  \FunctionTok{rename}\NormalTok{(}\AttributeTok{precio\_ponderado\_vivienda =}\NormalTok{ precio\_ponderado)}
\end{Highlighting}
\end{Shaded}

\begin{verbatim}
## New names:
## * `` -> `...1`
\end{verbatim}

\begin{Shaded}
\begin{Highlighting}[]
\NormalTok{pisos\_limpios }\OtherTok{\textless{}{-}} \FunctionTok{read\_csv}\NormalTok{(}\StringTok{\textquotesingle{}pisos\_limpios.csv\textquotesingle{}}\NormalTok{, }\AttributeTok{show\_col\_types =} \ConstantTok{FALSE}\NormalTok{) }\SpecialCharTok{\%\textgreater{}\%}
  \FunctionTok{rename}\NormalTok{(}\AttributeTok{barrio =}\NormalTok{ neighbourhood\_cleansed) }
\end{Highlighting}
\end{Shaded}

\begin{verbatim}
## New names:
## * `` -> `...1`
\end{verbatim}

\begin{Shaded}
\begin{Highlighting}[]
\CommentTok{\# Cargar archivo barrios con precios medios}
\NormalTok{barrios }\OtherTok{\textless{}{-}} \FunctionTok{read\_csv}\NormalTok{(}\StringTok{"barrios\_cluster\_variables.csv"}\NormalTok{, }\AttributeTok{show\_col\_types =} \ConstantTok{FALSE}\NormalTok{) }\SpecialCharTok{\%\textgreater{}\%}
  \FunctionTok{rename}\NormalTok{(}\AttributeTok{barrio =}\NormalTok{ neighbourhood\_cleansed) }\SpecialCharTok{\%\textgreater{}\%}
  \FunctionTok{select}\NormalTok{(barrio, precio\_ponderado\_vivienda\_media, precio\_ponderado\_barrio\_media)}

  \CommentTok{\#Cargar datos reseñas}
\NormalTok{reseñas }\OtherTok{\textless{}{-}} \FunctionTok{read\_csv}\NormalTok{(}\StringTok{"clustersResenas.csv"}\NormalTok{, }\AttributeTok{show\_col\_types =} \ConstantTok{FALSE}\NormalTok{) }\SpecialCharTok{\%\textgreater{}\%}
  \FunctionTok{rename}\NormalTok{(}\AttributeTok{cluster\_text =} \DecValTok{1}\NormalTok{) }\SpecialCharTok{\%\textgreater{}\%}
  \FunctionTok{rename}\NormalTok{(puntuacion\_media\_reseñas}\AttributeTok{\_ =}\NormalTok{ puntuacion\_media) }\SpecialCharTok{\%\textgreater{}\%}
  \FunctionTok{rename}\NormalTok{(reseñas}\AttributeTok{\_media\_normalizada\_cluster =}\NormalTok{ puntuacion\_media\_normal) }\SpecialCharTok{\%\textgreater{}\%}
  \FunctionTok{mutate}\NormalTok{(}\AttributeTok{cluster =} \FunctionTok{str\_extract}\NormalTok{(cluster\_text, }\StringTok{"}\SpecialCharTok{\textbackslash{}\textbackslash{}}\StringTok{d+"}\NormalTok{)) }\SpecialCharTok{\%\textgreater{}\%}   \CommentTok{\# Extrae "1", "2", etc.}
  \FunctionTok{select}\NormalTok{(}\SpecialCharTok{{-}}\NormalTok{cluster\_text)}
\end{Highlighting}
\end{Shaded}

\begin{verbatim}
## New names:
## * `` -> `...1`
\end{verbatim}

\begin{Shaded}
\begin{Highlighting}[]
\CommentTok{\# Cargar y añadir cluster a cada uno}
\NormalTok{cluster1 }\OtherTok{\textless{}{-}} \FunctionTok{read\_csv}\NormalTok{(}\StringTok{"cluster1.csv"}\NormalTok{, }\AttributeTok{show\_col\_types =} \ConstantTok{FALSE}\NormalTok{) }\SpecialCharTok{\%\textgreater{}\%} \FunctionTok{mutate}\NormalTok{(}\AttributeTok{cluster =} \DecValTok{1}\NormalTok{)}
\end{Highlighting}
\end{Shaded}

\begin{verbatim}
## New names:
## * `` -> `...1`
\end{verbatim}

\begin{Shaded}
\begin{Highlighting}[]
\NormalTok{cluster2 }\OtherTok{\textless{}{-}} \FunctionTok{read\_csv}\NormalTok{(}\StringTok{"cluster2.csv"}\NormalTok{, }\AttributeTok{show\_col\_types =} \ConstantTok{FALSE}\NormalTok{) }\SpecialCharTok{\%\textgreater{}\%} \FunctionTok{mutate}\NormalTok{(}\AttributeTok{cluster =} \DecValTok{2}\NormalTok{)}
\end{Highlighting}
\end{Shaded}

\begin{verbatim}
## New names:
## * `` -> `...1`
\end{verbatim}

\begin{Shaded}
\begin{Highlighting}[]
\NormalTok{cluster3 }\OtherTok{\textless{}{-}} \FunctionTok{read\_csv}\NormalTok{(}\StringTok{"cluster3.csv"}\NormalTok{, }\AttributeTok{show\_col\_types =} \ConstantTok{FALSE}\NormalTok{) }\SpecialCharTok{\%\textgreater{}\%} \FunctionTok{mutate}\NormalTok{(}\AttributeTok{cluster =} \DecValTok{3}\NormalTok{)}
\end{Highlighting}
\end{Shaded}

\begin{verbatim}
## New names:
## * `` -> `...1`
\end{verbatim}

\begin{Shaded}
\begin{Highlighting}[]
\NormalTok{cluster4 }\OtherTok{\textless{}{-}} \FunctionTok{read\_csv}\NormalTok{(}\StringTok{"cluster4.csv"}\NormalTok{, }\AttributeTok{show\_col\_types =} \ConstantTok{FALSE}\NormalTok{) }\SpecialCharTok{\%\textgreater{}\%} \FunctionTok{mutate}\NormalTok{(}\AttributeTok{cluster =} \DecValTok{4}\NormalTok{)}
\end{Highlighting}
\end{Shaded}

\begin{verbatim}
## New names:
## * `` -> `...1`
\end{verbatim}

\begin{Shaded}
\begin{Highlighting}[]
\CommentTok{\# Unir todos los clusters}
\NormalTok{clusters }\OtherTok{\textless{}{-}} \FunctionTok{bind\_rows}\NormalTok{(cluster1, cluster2, cluster3, cluster4)}

\CommentTok{\# Unir clusters con pisos}
\NormalTok{pisos\_cluster }\OtherTok{\textless{}{-}}\NormalTok{ pisos\_limpios }\SpecialCharTok{\%\textgreater{}\%}
  \FunctionTok{left\_join}\NormalTok{(clusters }\SpecialCharTok{\%\textgreater{}\%} \FunctionTok{select}\NormalTok{(barrio, cluster), }\AttributeTok{by =} \StringTok{"barrio"}\NormalTok{) }\SpecialCharTok{\%\textgreater{}\%}
  \FunctionTok{mutate}\NormalTok{(}\AttributeTok{cluster =} \FunctionTok{as.character}\NormalTok{(cluster))}

\CommentTok{\# Unir reseñas con pisos }
\NormalTok{pisos\_reseñas }\OtherTok{\textless{}{-}}\NormalTok{ pisos\_cluster }\SpecialCharTok{\%\textgreater{}\%}
  \FunctionTok{left\_join}\NormalTok{(reseñas, }\AttributeTok{by =} \StringTok{"cluster"}\NormalTok{)}

\CommentTok{\# Unir al dataset de viviendas}
\NormalTok{pisos\_final }\OtherTok{\textless{}{-}}\NormalTok{ pisos\_reseñas }\SpecialCharTok{\%\textgreater{}\%}
  \FunctionTok{left\_join}\NormalTok{(barrios, }\AttributeTok{by =} \StringTok{"barrio"}\NormalTok{)}

\NormalTok{pisos\_final }\OtherTok{\textless{}{-}}\NormalTok{ pisos\_final }\SpecialCharTok{\%\textgreater{}\%} \FunctionTok{select}\NormalTok{(}\SpecialCharTok{{-}}\DecValTok{1}\NormalTok{)}

\NormalTok{pisos\_final }\OtherTok{\textless{}{-}}\NormalTok{ pisos\_final }\SpecialCharTok{\%\textgreater{}\%}
  \FunctionTok{left\_join}\NormalTok{(pisos\_fiabilidad }\SpecialCharTok{\%\textgreater{}\%}
      \FunctionTok{select}\NormalTok{(id, score\_fiabilidad, fiabilidad\_normalizada, score\_fiabilidad\_0a5, precio\_ponderado\_barrio, precio\_ponderado\_vivienda),}
    \AttributeTok{by =} \FunctionTok{c}\NormalTok{(}\StringTok{"listing\_id"} \OtherTok{=} \StringTok{"id"}\NormalTok{)}
\NormalTok{  )}

\CommentTok{\# Unir al dataset de viviendas}
\NormalTok{pisos\_final }\OtherTok{\textless{}{-}}\NormalTok{ pisos\_final }\SpecialCharTok{\%\textgreater{}\%}
  \FunctionTok{left\_join}\NormalTok{(barrios\_reseñas, }\AttributeTok{by =} \StringTok{"barrio"}\NormalTok{)}

\NormalTok{pisos\_final }\OtherTok{\textless{}{-}}\NormalTok{ pisos\_final }\SpecialCharTok{\%\textgreater{}\%} \FunctionTok{select}\NormalTok{(}\SpecialCharTok{{-}}\DecValTok{1}\NormalTok{)}
\end{Highlighting}
\end{Shaded}

\begin{Shaded}
\begin{Highlighting}[]
\CommentTok{\# Filtrar columnas clave y eliminar filas incompletas}
\NormalTok{datos\_modelo }\OtherTok{\textless{}{-}}\NormalTok{ pisos\_final }\SpecialCharTok{\%\textgreater{}\%}
  \FunctionTok{select}\NormalTok{(price, }\CommentTok{\# Variable objetivo (precio real)}
\NormalTok{    bedrooms, bathrooms, accommodates,     }\CommentTok{\# Variables físicas}
\NormalTok{    precio\_ponderado\_barrio\_media, barrio,}
\NormalTok{    precio\_ponderado\_vivienda\_media, precio\_ponderado\_barrio, precio\_ponderado\_vivienda,}
\NormalTok{    fiabilidad\_normalizada, reseñas\_media\_normalizada\_cluster, puntuacion\_media\_normal\_reseñas,}
\NormalTok{    cluster}
\NormalTok{  ) }\SpecialCharTok{\%\textgreater{}\%}
  \FunctionTok{filter}\NormalTok{(}\FunctionTok{complete.cases}\NormalTok{(.)) }\SpecialCharTok{\%\textgreater{}\%}  \CommentTok{\# Elimina filas con valores faltantes}
  \FunctionTok{mutate}\NormalTok{(}\AttributeTok{cluster =} \FunctionTok{as.factor}\NormalTok{(cluster))  }\CommentTok{\# Convierte a factor para modelos}

\CommentTok{\# Eliminar outliers extremos (5\% inferior y 5\% superior del precio)}
\NormalTok{cuantiles }\OtherTok{\textless{}{-}} \FunctionTok{quantile}\NormalTok{(datos\_modelo}\SpecialCharTok{$}\NormalTok{price, }\AttributeTok{probs =} \FunctionTok{c}\NormalTok{(}\FloatTok{0.05}\NormalTok{, }\FloatTok{0.95}\NormalTok{), }\AttributeTok{na.rm =} \ConstantTok{TRUE}\NormalTok{)}

\NormalTok{datos\_modelo }\OtherTok{\textless{}{-}}\NormalTok{ datos\_modelo }\SpecialCharTok{\%\textgreater{}\%}
  \FunctionTok{filter}\NormalTok{(price }\SpecialCharTok{\textgreater{}=}\NormalTok{ cuantiles[}\DecValTok{1}\NormalTok{], price }\SpecialCharTok{\textless{}=}\NormalTok{ cuantiles[}\DecValTok{2}\NormalTok{])}
\end{Highlighting}
\end{Shaded}

\begin{Shaded}
\begin{Highlighting}[]
\CommentTok{\# Añadir log\_precio}
\NormalTok{datos\_modelo }\OtherTok{\textless{}{-}}\NormalTok{ datos\_modelo }\SpecialCharTok{\%\textgreater{}\%}
  \FunctionTok{mutate}\NormalTok{(}\AttributeTok{log\_precio =} \FunctionTok{log}\NormalTok{(price }\SpecialCharTok{+} \DecValTok{1}\NormalTok{))}

\CommentTok{\# Dividir en train/test (80/20)}
\FunctionTok{set.seed}\NormalTok{(}\DecValTok{123}\NormalTok{)}
\NormalTok{n }\OtherTok{\textless{}{-}} \FunctionTok{nrow}\NormalTok{(datos\_modelo)}
\NormalTok{train\_indices }\OtherTok{\textless{}{-}} \FunctionTok{sample}\NormalTok{(}\DecValTok{1}\SpecialCharTok{:}\NormalTok{n, }\AttributeTok{size =} \FloatTok{0.8} \SpecialCharTok{*}\NormalTok{ n)}

\NormalTok{train }\OtherTok{\textless{}{-}}\NormalTok{ datos\_modelo[train\_indices, ]}
\NormalTok{test }\OtherTok{\textless{}{-}}\NormalTok{ datos\_modelo[}\SpecialCharTok{{-}}\NormalTok{train\_indices, ]}
\end{Highlighting}
\end{Shaded}

Se transformó en log para suavizar el impacto de los valores extremos,
mejorar la relación lineal entre variables predictoras y el objetivo y
facilitar una predicción más precisa en el rango central de precios.

\begin{Shaded}
\begin{Highlighting}[]
\CommentTok{\# Eliminar columnas no numéricas y preparar matrices}
\NormalTok{x\_vars }\OtherTok{\textless{}{-}} \FunctionTok{c}\NormalTok{(}
  \StringTok{"bedrooms"}\NormalTok{, }\StringTok{"bathrooms"}\NormalTok{, }\StringTok{"accommodates"}\NormalTok{,}
  \StringTok{"precio\_ponderado\_vivienda\_media"}\NormalTok{, }\StringTok{"precio\_ponderado\_barrio\_media"}\NormalTok{,}
  \StringTok{"fiabilidad\_normalizada"}\NormalTok{, }\StringTok{"precio\_ponderado\_vivienda"}\NormalTok{,}
  \StringTok{"puntuacion\_media\_normal\_reseñas"}\NormalTok{)}

\CommentTok{\# PERO mantener cluster y barrio en el dataframe}
\NormalTok{train}\SpecialCharTok{$}\NormalTok{cluster }\OtherTok{\textless{}{-}}\NormalTok{ datos\_modelo[train\_indices, }\StringTok{"cluster"}\NormalTok{]}
\NormalTok{test}\SpecialCharTok{$}\NormalTok{cluster }\OtherTok{\textless{}{-}}\NormalTok{ datos\_modelo}\SpecialCharTok{$}\NormalTok{cluster[}\SpecialCharTok{{-}}\NormalTok{train\_indices]}
\NormalTok{train}\SpecialCharTok{$}\NormalTok{barrio }\OtherTok{\textless{}{-}}\NormalTok{ datos\_modelo[train\_indices, }\StringTok{"barrio"}\NormalTok{]}
\NormalTok{test}\SpecialCharTok{$}\NormalTok{barrio }\OtherTok{\textless{}{-}}\NormalTok{ datos\_modelo[}\SpecialCharTok{{-}}\NormalTok{train\_indices, }\StringTok{"barrio"}\NormalTok{]}

\NormalTok{X\_train }\OtherTok{\textless{}{-}} \FunctionTok{as.matrix}\NormalTok{(}\FunctionTok{sapply}\NormalTok{(train[, x\_vars], as.numeric))}
\NormalTok{X\_test }\OtherTok{\textless{}{-}} \FunctionTok{as.matrix}\NormalTok{(}\FunctionTok{sapply}\NormalTok{(test[, x\_vars], as.numeric))}
\NormalTok{y\_train }\OtherTok{\textless{}{-}}\NormalTok{ train}\SpecialCharTok{$}\NormalTok{log\_precio}
\NormalTok{y\_test }\OtherTok{\textless{}{-}}\NormalTok{ test}\SpecialCharTok{$}\NormalTok{log\_precio}
\end{Highlighting}
\end{Shaded}

\begin{Shaded}
\begin{Highlighting}[]
\CommentTok{\# Modelo de regresión lineal}
\NormalTok{modelo\_lm }\OtherTok{\textless{}{-}} \FunctionTok{lm}\NormalTok{(log\_precio }\SpecialCharTok{\textasciitilde{}}\NormalTok{ bedrooms }\SpecialCharTok{+}\NormalTok{ bathrooms }\SpecialCharTok{+}\NormalTok{ accommodates }\SpecialCharTok{+}
\NormalTok{                  precio\_ponderado\_vivienda\_media }\SpecialCharTok{+}\NormalTok{ precio\_ponderado\_barrio\_media }\SpecialCharTok{+}
\NormalTok{                  fiabilidad\_normalizada }\SpecialCharTok{+}\NormalTok{ precio\_ponderado\_vivienda }\SpecialCharTok{+}\NormalTok{ precio\_ponderado\_barrio }\SpecialCharTok{+}
\NormalTok{                  puntuacion\_media\_normal\_reseñas }\SpecialCharTok{+}
\NormalTok{                  reseñas\_media\_normalizada\_cluster,}
                \AttributeTok{data =}\NormalTok{ train)}

\CommentTok{\# Predicciones}
\NormalTok{pred\_log\_lm }\OtherTok{\textless{}{-}} \FunctionTok{predict}\NormalTok{(modelo\_lm, }\AttributeTok{newdata =}\NormalTok{ test)}
\NormalTok{pred\_lm }\OtherTok{\textless{}{-}} \FunctionTok{exp}\NormalTok{(pred\_log\_lm) }\SpecialCharTok{{-}} \DecValTok{1}
\NormalTok{real }\OtherTok{\textless{}{-}} \FunctionTok{exp}\NormalTok{(test}\SpecialCharTok{$}\NormalTok{log\_precio) }\SpecialCharTok{{-}} \DecValTok{1}

\CommentTok{\# Evaluación}
\FunctionTok{data.frame}\NormalTok{(}
  \AttributeTok{Modelo =} \StringTok{"Regresión Lineal"}\NormalTok{,}
  \AttributeTok{RMSE =} \FunctionTok{rmse}\NormalTok{(real, pred\_lm),}
  \AttributeTok{MAE =} \FunctionTok{mae}\NormalTok{(real, pred\_lm),}
  \AttributeTok{R2 =} \FunctionTok{cor}\NormalTok{(real, pred\_lm)}\SpecialCharTok{\^{}}\DecValTok{2}
\NormalTok{)}
\end{Highlighting}
\end{Shaded}

\begin{verbatim}
##             Modelo     RMSE      MAE        R2
## 1 Regresión Lineal 32.04329 23.49962 0.2210106
\end{verbatim}

\begin{Shaded}
\begin{Highlighting}[]
\CommentTok{\# Entrenar random forest}
\NormalTok{modelo\_rf }\OtherTok{\textless{}{-}} \FunctionTok{randomForest}\NormalTok{(}
  \AttributeTok{formula =}\NormalTok{ log\_precio }\SpecialCharTok{\textasciitilde{}}\NormalTok{ bedrooms }\SpecialCharTok{+}\NormalTok{ bathrooms }\SpecialCharTok{+}\NormalTok{ accommodates }\SpecialCharTok{+}
\NormalTok{    precio\_ponderado\_vivienda\_media }\SpecialCharTok{+}\NormalTok{ precio\_ponderado\_barrio\_media }\SpecialCharTok{+}
\NormalTok{    fiabilidad\_normalizada }\SpecialCharTok{+}\NormalTok{ precio\_ponderado\_vivienda }\SpecialCharTok{+}
\NormalTok{    puntuacion\_media\_normal\_reseñas,}
  \AttributeTok{data =}\NormalTok{ train,}
  \AttributeTok{ntree =} \DecValTok{500}\NormalTok{,}
  \AttributeTok{importance =} \ConstantTok{TRUE}
\NormalTok{)}

\CommentTok{\# Predicciones}
\NormalTok{pred\_log\_rf }\OtherTok{\textless{}{-}} \FunctionTok{predict}\NormalTok{(modelo\_rf, }\AttributeTok{newdata =}\NormalTok{ test)}
\NormalTok{pred\_rf }\OtherTok{\textless{}{-}} \FunctionTok{exp}\NormalTok{(pred\_log\_rf) }\SpecialCharTok{{-}} \DecValTok{1}

\CommentTok{\# Evaluación}
\FunctionTok{data.frame}\NormalTok{(}
  \AttributeTok{Modelo =} \StringTok{"Random Forest"}\NormalTok{,}
  \AttributeTok{RMSE =} \FunctionTok{rmse}\NormalTok{(real, pred\_rf),}
  \AttributeTok{MAE =} \FunctionTok{mae}\NormalTok{(real, pred\_rf),}
  \AttributeTok{R2 =} \FunctionTok{cor}\NormalTok{(real, pred\_rf)}\SpecialCharTok{\^{}}\DecValTok{2}
\NormalTok{)}
\end{Highlighting}
\end{Shaded}

\begin{verbatim}
##          Modelo     RMSE      MAE        R2
## 1 Random Forest 31.66033 23.10396 0.2309169
\end{verbatim}

\begin{Shaded}
\begin{Highlighting}[]
\NormalTok{modelo\_xgb }\OtherTok{\textless{}{-}} \FunctionTok{xgboost}\NormalTok{(}
  \AttributeTok{data =}\NormalTok{ X\_train,}
  \AttributeTok{label =}\NormalTok{ y\_train,}
  \AttributeTok{nrounds =} \DecValTok{100}\NormalTok{,}
  \AttributeTok{max\_depth =} \DecValTok{4}\NormalTok{,}
  \AttributeTok{eta =} \FloatTok{0.1}\NormalTok{,}
  \AttributeTok{objective =} \StringTok{"reg:squarederror"}\NormalTok{,}
  \AttributeTok{verbose =} \DecValTok{0}
\NormalTok{)}

\CommentTok{\# Predicción}
\NormalTok{pred\_log }\OtherTok{\textless{}{-}} \FunctionTok{predict}\NormalTok{(modelo\_xgb, X\_test)}
\NormalTok{pred }\OtherTok{\textless{}{-}} \FunctionTok{exp}\NormalTok{(pred\_log) }\SpecialCharTok{{-}} \DecValTok{1}
\NormalTok{real }\OtherTok{\textless{}{-}} \FunctionTok{exp}\NormalTok{(y\_test) }\SpecialCharTok{{-}} \DecValTok{1}

\CommentTok{\# Métricas}
\FunctionTok{data.frame}\NormalTok{(}
  \AttributeTok{RMSE =} \FunctionTok{rmse}\NormalTok{(real, pred),}
  \AttributeTok{MAE =} \FunctionTok{mae}\NormalTok{(real, pred),}
  \AttributeTok{R2 =} \FunctionTok{cor}\NormalTok{(real, pred)}\SpecialCharTok{\^{}}\DecValTok{2}
\NormalTok{)}
\end{Highlighting}
\end{Shaded}

\begin{verbatim}
##       RMSE      MAE        R2
## 1 31.07777 22.97804 0.2701054
\end{verbatim}

\begin{Shaded}
\begin{Highlighting}[]
\CommentTok{\# Reutilizamos train y test definidos antes}
\NormalTok{real }\OtherTok{\textless{}{-}} \FunctionTok{exp}\NormalTok{(test}\SpecialCharTok{$}\NormalTok{log\_precio) }\SpecialCharTok{{-}} \DecValTok{1}

\CommentTok{\# Modelo lineal}
\NormalTok{modelo\_lm }\OtherTok{\textless{}{-}} \FunctionTok{lm}\NormalTok{(log\_precio }\SpecialCharTok{\textasciitilde{}}\NormalTok{ ., }\AttributeTok{data =}\NormalTok{ train[, }\FunctionTok{c}\NormalTok{(x\_vars, }\StringTok{"log\_precio"}\NormalTok{)])}
\NormalTok{pred\_lm }\OtherTok{\textless{}{-}} \FunctionTok{exp}\NormalTok{(}\FunctionTok{predict}\NormalTok{(modelo\_lm, }\AttributeTok{newdata =}\NormalTok{ test)) }\SpecialCharTok{{-}} \DecValTok{1}

\CommentTok{\# Modelo Random Forest}
\NormalTok{modelo\_rf }\OtherTok{\textless{}{-}} \FunctionTok{randomForest}\NormalTok{(}
  \AttributeTok{formula =}\NormalTok{ log\_precio }\SpecialCharTok{\textasciitilde{}}\NormalTok{ .,}
  \AttributeTok{data =}\NormalTok{ train[, }\FunctionTok{c}\NormalTok{(x\_vars, }\StringTok{"log\_precio"}\NormalTok{)],}
  \AttributeTok{ntree =} \DecValTok{500}\NormalTok{,}
  \AttributeTok{importance =} \ConstantTok{TRUE}
\NormalTok{)}
\NormalTok{pred\_rf }\OtherTok{\textless{}{-}} \FunctionTok{exp}\NormalTok{(}\FunctionTok{predict}\NormalTok{(modelo\_rf, }\AttributeTok{newdata =}\NormalTok{ test)) }\SpecialCharTok{{-}} \DecValTok{1}

\CommentTok{\# Ya tienes modelo\_xgb y pred\_xgb = pred}

\CommentTok{\# Guardar resultados}
\NormalTok{comparacion\_modelos }\OtherTok{\textless{}{-}} \FunctionTok{bind\_rows}\NormalTok{(}
  \FunctionTok{tibble}\NormalTok{(}\AttributeTok{Modelo =} \StringTok{"XGBoost"}\NormalTok{, }\AttributeTok{RMSE =} \FunctionTok{rmse}\NormalTok{(real, pred), }\AttributeTok{MAE =} \FunctionTok{mae}\NormalTok{(real, pred), }\AttributeTok{R2 =} \FunctionTok{cor}\NormalTok{(real, pred)}\SpecialCharTok{\^{}}\DecValTok{2}\NormalTok{),}
  \FunctionTok{tibble}\NormalTok{(}\AttributeTok{Modelo =} \StringTok{"Random Forest"}\NormalTok{, }\AttributeTok{RMSE =} \FunctionTok{rmse}\NormalTok{(real, pred\_rf), }\AttributeTok{MAE =} \FunctionTok{mae}\NormalTok{(real, pred\_rf), }\AttributeTok{R2 =} \FunctionTok{cor}\NormalTok{(real, pred\_rf)}\SpecialCharTok{\^{}}\DecValTok{2}\NormalTok{),}
  \FunctionTok{tibble}\NormalTok{(}\AttributeTok{Modelo =} \StringTok{"Regresión Lineal"}\NormalTok{, }\AttributeTok{RMSE =} \FunctionTok{rmse}\NormalTok{(real, pred\_lm), }\AttributeTok{MAE =} \FunctionTok{mae}\NormalTok{(real, pred\_lm), }\AttributeTok{R2 =} \FunctionTok{cor}\NormalTok{(real, pred\_lm)}\SpecialCharTok{\^{}}\DecValTok{2}\NormalTok{)}
\NormalTok{)}
\end{Highlighting}
\end{Shaded}

\begin{Shaded}
\begin{Highlighting}[]
\NormalTok{comparacion\_modelos }\SpecialCharTok{\%\textgreater{}\%}
  \FunctionTok{kable}\NormalTok{(}\AttributeTok{digits =} \DecValTok{2}\NormalTok{) }\SpecialCharTok{\%\textgreater{}\%}
  \FunctionTok{kable\_styling}\NormalTok{(}\AttributeTok{full\_width =} \ConstantTok{FALSE}\NormalTok{)}
\end{Highlighting}
\end{Shaded}

\begin{longtable}[t]{lrrr}
\toprule
Modelo & RMSE & MAE & R2\\
\midrule
XGBoost & 31.08 & 22.98 & 0.27\\
Random Forest & 31.68 & 23.06 & 0.23\\
Regresión Lineal & 32.49 & 24.11 & 0.20\\
\bottomrule
\end{longtable}

\begin{Shaded}
\begin{Highlighting}[]
\CommentTok{\# Normalizar cada métrica a escala 0–1 para comparación visual}
\NormalTok{comparacion\_norm }\OtherTok{\textless{}{-}}\NormalTok{ comparacion\_modelos }\SpecialCharTok{\%\textgreater{}\%}
  \FunctionTok{mutate}\NormalTok{(}
    \AttributeTok{RMSE\_norm =}\NormalTok{ (RMSE }\SpecialCharTok{{-}} \FunctionTok{min}\NormalTok{(RMSE)) }\SpecialCharTok{/}\NormalTok{ (}\FunctionTok{max}\NormalTok{(RMSE) }\SpecialCharTok{{-}} \FunctionTok{min}\NormalTok{(RMSE)),}
    \AttributeTok{MAE\_norm =}\NormalTok{ (MAE }\SpecialCharTok{{-}} \FunctionTok{min}\NormalTok{(MAE)) }\SpecialCharTok{/}\NormalTok{ (}\FunctionTok{max}\NormalTok{(MAE) }\SpecialCharTok{{-}} \FunctionTok{min}\NormalTok{(MAE)),}
    \AttributeTok{R2\_norm =}\NormalTok{ (R2 }\SpecialCharTok{{-}} \FunctionTok{min}\NormalTok{(R2)) }\SpecialCharTok{/}\NormalTok{ (}\FunctionTok{max}\NormalTok{(R2) }\SpecialCharTok{{-}} \FunctionTok{min}\NormalTok{(R2))}
\NormalTok{  ) }\SpecialCharTok{\%\textgreater{}\%}
  \FunctionTok{select}\NormalTok{(Modelo, }\AttributeTok{RMSE =}\NormalTok{ RMSE\_norm, }\AttributeTok{MAE =}\NormalTok{ MAE\_norm, }\AttributeTok{R2 =}\NormalTok{ R2\_norm) }\SpecialCharTok{\%\textgreater{}\%}
  \FunctionTok{pivot\_longer}\NormalTok{(}\AttributeTok{cols =} \FunctionTok{c}\NormalTok{(RMSE, MAE, R2), }\AttributeTok{names\_to =} \StringTok{"Métrica"}\NormalTok{, }\AttributeTok{values\_to =} \StringTok{"Valor"}\NormalTok{)}

\CommentTok{\# Gráfico comparativo proporcional}
\FunctionTok{ggplot}\NormalTok{(comparacion\_norm, }\FunctionTok{aes}\NormalTok{(}\AttributeTok{x =}\NormalTok{ Modelo, }\AttributeTok{y =}\NormalTok{ Valor, }\AttributeTok{fill =}\NormalTok{ Métrica)) }\SpecialCharTok{+}
  \FunctionTok{geom\_col}\NormalTok{(}\AttributeTok{position =} \StringTok{"dodge"}\NormalTok{) }\SpecialCharTok{+}
  \FunctionTok{labs}\NormalTok{(}
    \AttributeTok{title =} \StringTok{"Comparación de modelos (normalizado 0–1)"}\NormalTok{,}
    \AttributeTok{y =} \StringTok{"Valor normalizado (0–1)"}\NormalTok{,}
    \AttributeTok{x =} \StringTok{""}
\NormalTok{  ) }\SpecialCharTok{+}
  \FunctionTok{theme\_minimal}\NormalTok{(}\AttributeTok{base\_size =} \DecValTok{14}\NormalTok{) }\SpecialCharTok{+}
  \FunctionTok{scale\_fill\_brewer}\NormalTok{(}\AttributeTok{palette =} \StringTok{"Set2"}\NormalTok{)}
\end{Highlighting}
\end{Shaded}

\includegraphics{Predicción_precio_vivienda_files/figure-latex/Comparación modelos-1.pdf}

XGBoost se comporta como el modelo más completo: explica mejor la
variabilidad del precio y mantiene un error razonablemente bajo. Random
Forest ofrece predicciones algo más conservadoras y precisas en euros,
pero con menos poder explicativo. La regresión lineal no capta bien las
complejidades del precio de vivienda

\begin{Shaded}
\begin{Highlighting}[]
\CommentTok{\# Calcular importancia}
\NormalTok{importance\_matrix }\OtherTok{\textless{}{-}} \FunctionTok{xgb.importance}\NormalTok{(}\AttributeTok{model =}\NormalTok{ modelo\_xgb)}

\CommentTok{\# Mostrar como tabla}
\NormalTok{importance\_matrix }\SpecialCharTok{\%\textgreater{}\%}
  \FunctionTok{kable}\NormalTok{() }\SpecialCharTok{\%\textgreater{}\%}
  \FunctionTok{kable\_styling}\NormalTok{(}\AttributeTok{full\_width =} \ConstantTok{FALSE}\NormalTok{, }\AttributeTok{bootstrap\_options =} \FunctionTok{c}\NormalTok{(}\StringTok{"striped"}\NormalTok{, }\StringTok{"hover"}\NormalTok{))}
\end{Highlighting}
\end{Shaded}

\begin{longtable}[t]{lrrr}
\toprule
Feature & Gain & Cover & Frequency\\
\midrule
precio\_ponderado\_vivienda & 0.3352186 & 0.1304385 & 0.1256831\\
fiabilidad\_normalizada & 0.3020720 & 0.4243725 & 0.4034608\\
puntuacion\_media\_normal\_reseñas & 0.0818500 & 0.1366450 & 0.1220401\\
bedrooms & 0.0723520 & 0.0284534 & 0.0610200\\
precio\_ponderado\_barrio\_media & 0.0673975 & 0.1231031 & 0.0810565\\
\addlinespace
bathrooms & 0.0635414 & 0.0259743 & 0.0537341\\
precio\_ponderado\_vivienda\_media & 0.0590802 & 0.1030143 & 0.1092896\\
accommodates & 0.0184883 & 0.0279990 & 0.0437158\\
\bottomrule
\end{longtable}

\begin{Shaded}
\begin{Highlighting}[]
\CommentTok{\# Gráfico de importancia}
\FunctionTok{xgb.plot.importance}\NormalTok{(importance\_matrix, }\AttributeTok{top\_n =} \DecValTok{10}\NormalTok{, }\AttributeTok{rel\_to\_first =} \ConstantTok{TRUE}\NormalTok{,}
                    \AttributeTok{xlab =} \StringTok{"Importancia relativa"}\NormalTok{)}
\end{Highlighting}
\end{Shaded}

\includegraphics{Predicción_precio_vivienda_files/figure-latex/Importancia variables en modelo-1.pdf}

\begin{Shaded}
\begin{Highlighting}[]
\CommentTok{\# Crear dataframe con resultados}
\NormalTok{resultados }\OtherTok{\textless{}{-}}\NormalTok{ test }\SpecialCharTok{\%\textgreater{}\%}
  \FunctionTok{mutate}\NormalTok{(}
    \AttributeTok{precio\_real =} \FunctionTok{exp}\NormalTok{(log\_precio) }\SpecialCharTok{{-}} \DecValTok{1}\NormalTok{,}
    \AttributeTok{precio\_predicho =}\NormalTok{ pred}
\NormalTok{  )}

\CommentTok{\# Gráfico}
\FunctionTok{ggplot}\NormalTok{(resultados, }\FunctionTok{aes}\NormalTok{(}\AttributeTok{x =}\NormalTok{ precio\_real, }\AttributeTok{y =}\NormalTok{ precio\_predicho)) }\SpecialCharTok{+}
  \FunctionTok{geom\_point}\NormalTok{(}\AttributeTok{alpha =} \FloatTok{0.4}\NormalTok{, }\AttributeTok{color =} \StringTok{"\#1f78b4"}\NormalTok{, }\AttributeTok{size =} \DecValTok{2}\NormalTok{) }\SpecialCharTok{+}
  \FunctionTok{geom\_abline}\NormalTok{(}\AttributeTok{slope =} \DecValTok{1}\NormalTok{, }\AttributeTok{intercept =} \DecValTok{0}\NormalTok{, }\AttributeTok{color =} \StringTok{"red"}\NormalTok{, }\AttributeTok{linetype =} \StringTok{"dashed"}\NormalTok{) }\SpecialCharTok{+}
  \FunctionTok{labs}\NormalTok{(}
    \AttributeTok{title =} \StringTok{"Precio Real vs Precio Predicho"}\NormalTok{,}
    \AttributeTok{x =} \StringTok{"Precio Real (€)"}\NormalTok{,}
    \AttributeTok{y =} \StringTok{"Precio Predicho (€)"}
\NormalTok{  ) }\SpecialCharTok{+}
  \FunctionTok{theme\_minimal}\NormalTok{(}\AttributeTok{base\_size =} \DecValTok{14}\NormalTok{) }\SpecialCharTok{+}
  \FunctionTok{annotate}\NormalTok{(}\StringTok{"text"}\NormalTok{, }\AttributeTok{x =} \FunctionTok{min}\NormalTok{(resultados}\SpecialCharTok{$}\NormalTok{precio\_real), }\AttributeTok{y =} \FunctionTok{max}\NormalTok{(resultados}\SpecialCharTok{$}\NormalTok{precio\_predicho),}
           \AttributeTok{label =} \FunctionTok{paste}\NormalTok{(}\StringTok{"R² ="}\NormalTok{, }\FunctionTok{round}\NormalTok{(}\FunctionTok{cor}\NormalTok{(resultados}\SpecialCharTok{$}\NormalTok{precio\_real, resultados}\SpecialCharTok{$}\NormalTok{precio\_predicho)}\SpecialCharTok{\^{}}\DecValTok{2}\NormalTok{, }\DecValTok{3}\NormalTok{)),}
           \AttributeTok{hjust =} \DecValTok{0}\NormalTok{, }\AttributeTok{color =} \StringTok{"darkgreen"}\NormalTok{)}
\end{Highlighting}
\end{Shaded}

\includegraphics{Predicción_precio_vivienda_files/figure-latex/Precio Real v Predicho-1.pdf}

\begin{Shaded}
\begin{Highlighting}[]
\NormalTok{resultados\_clusters }\OtherTok{\textless{}{-}}\NormalTok{ datos\_modelo }\SpecialCharTok{\%\textgreater{}\%}
  \FunctionTok{split}\NormalTok{(.}\SpecialCharTok{$}\NormalTok{cluster) }\SpecialCharTok{\%\textgreater{}\%}
  \FunctionTok{map\_df}\NormalTok{(}\ControlFlowTok{function}\NormalTok{(df) \{}
    \ControlFlowTok{if}\NormalTok{ (}\FunctionTok{nrow}\NormalTok{(df) }\SpecialCharTok{\textless{}} \DecValTok{20}\NormalTok{) }\FunctionTok{return}\NormalTok{(}\ConstantTok{NULL}\NormalTok{)  }\CommentTok{\# evitar errores con clusters pequeños}
    
    \FunctionTok{set.seed}\NormalTok{(}\DecValTok{123}\NormalTok{)}
\NormalTok{    train\_indices }\OtherTok{\textless{}{-}} \FunctionTok{sample}\NormalTok{(}\DecValTok{1}\SpecialCharTok{:}\FunctionTok{nrow}\NormalTok{(df), }\AttributeTok{size =} \FloatTok{0.8} \SpecialCharTok{*} \FunctionTok{nrow}\NormalTok{(df))}
\NormalTok{    train }\OtherTok{\textless{}{-}}\NormalTok{ df[train\_indices, ]}
\NormalTok{    test }\OtherTok{\textless{}{-}}\NormalTok{ df[}\SpecialCharTok{{-}}\NormalTok{train\_indices, ]}
    
\NormalTok{    x\_vars }\OtherTok{\textless{}{-}} \FunctionTok{c}\NormalTok{(}\StringTok{"bedrooms"}\NormalTok{, }\StringTok{"bathrooms"}\NormalTok{, }\StringTok{"accommodates"}\NormalTok{,}
                \StringTok{"precio\_ponderado\_barrio\_media"}\NormalTok{,}
                \StringTok{"fiabilidad\_normalizada"}\NormalTok{,}
                \StringTok{"puntuacion\_media\_normal\_reseñas"}\NormalTok{)}
    
\NormalTok{    X\_train }\OtherTok{\textless{}{-}} \FunctionTok{as.matrix}\NormalTok{(}\FunctionTok{sapply}\NormalTok{(train[, x\_vars], as.numeric))}
\NormalTok{    X\_test }\OtherTok{\textless{}{-}} \FunctionTok{as.matrix}\NormalTok{(}\FunctionTok{sapply}\NormalTok{(test[, x\_vars], as.numeric))}
\NormalTok{    y\_train }\OtherTok{\textless{}{-}} \FunctionTok{log}\NormalTok{(train}\SpecialCharTok{$}\NormalTok{price }\SpecialCharTok{+} \DecValTok{1}\NormalTok{)}
\NormalTok{    y\_test }\OtherTok{\textless{}{-}} \FunctionTok{log}\NormalTok{(test}\SpecialCharTok{$}\NormalTok{price }\SpecialCharTok{+} \DecValTok{1}\NormalTok{)}
    
\NormalTok{    modelo }\OtherTok{\textless{}{-}} \FunctionTok{xgboost}\NormalTok{(}\AttributeTok{data =}\NormalTok{ X\_train, }\AttributeTok{label =}\NormalTok{ y\_train, }\AttributeTok{nrounds =} \DecValTok{100}\NormalTok{,}
                      \AttributeTok{max\_depth =} \DecValTok{4}\NormalTok{, }\AttributeTok{eta =} \FloatTok{0.1}\NormalTok{, }\AttributeTok{objective =} \StringTok{"reg:squarederror"}\NormalTok{, }\AttributeTok{verbose =} \DecValTok{0}\NormalTok{)}
    
\NormalTok{    pred }\OtherTok{\textless{}{-}} \FunctionTok{exp}\NormalTok{(}\FunctionTok{predict}\NormalTok{(modelo, X\_test)) }\SpecialCharTok{{-}} \DecValTok{1}
\NormalTok{    real }\OtherTok{\textless{}{-}} \FunctionTok{exp}\NormalTok{(y\_test) }\SpecialCharTok{{-}} \DecValTok{1}
    
\NormalTok{    resultado }\OtherTok{\textless{}{-}} \FunctionTok{tibble}\NormalTok{(}
      \AttributeTok{cluster =} \FunctionTok{unique}\NormalTok{(df}\SpecialCharTok{$}\NormalTok{cluster),}
      \AttributeTok{RMSE =} \FunctionTok{rmse}\NormalTok{(real, pred),}
      \AttributeTok{MAE =} \FunctionTok{mae}\NormalTok{(real, pred),}
      \AttributeTok{R2 =} \FunctionTok{cor}\NormalTok{(real, pred)}\SpecialCharTok{\^{}}\DecValTok{2}
\NormalTok{    )}
    
    \FunctionTok{print}\NormalTok{(resultado)  }\CommentTok{\# Mostrar resultado por consola}
    \FunctionTok{return}\NormalTok{(resultado)}
\NormalTok{  \})}
\end{Highlighting}
\end{Shaded}

\begin{verbatim}
## # A tibble: 1 x 4
##   cluster  RMSE   MAE    R2
##   <fct>   <dbl> <dbl> <dbl>
## 1 1        31.5  24.6 0.285
## # A tibble: 1 x 4
##   cluster  RMSE   MAE    R2
##   <fct>   <dbl> <dbl> <dbl>
## 1 2        33.1  23.7 0.173
## # A tibble: 1 x 4
##   cluster  RMSE   MAE    R2
##   <fct>   <dbl> <dbl> <dbl>
## 1 3        36.6  27.4 0.190
## # A tibble: 1 x 4
##   cluster  RMSE   MAE    R2
##   <fct>   <dbl> <dbl> <dbl>
## 1 4        29.9  22.8 0.248
\end{verbatim}

\begin{Shaded}
\begin{Highlighting}[]
\CommentTok{\# Tabla final}
\NormalTok{resultados\_clusters }\SpecialCharTok{\%\textgreater{}\%}
  \FunctionTok{rename}\NormalTok{(}\StringTok{"Cluster"} \OtherTok{=}\NormalTok{ cluster, }\StringTok{"Error Medio (RMSE)"} \OtherTok{=}\NormalTok{ RMSE,}
         \StringTok{"Error Absoluto (MAE)"} \OtherTok{=}\NormalTok{ MAE, }\StringTok{"Variabilidad Explicada (R²)"} \OtherTok{=}\NormalTok{ R2) }\SpecialCharTok{\%\textgreater{}\%}
  \FunctionTok{kable}\NormalTok{() }\SpecialCharTok{\%\textgreater{}\%}
  \FunctionTok{kable\_styling}\NormalTok{(}\AttributeTok{full\_width =} \ConstantTok{FALSE}\NormalTok{)}
\end{Highlighting}
\end{Shaded}

\begin{longtable}[t]{lrrr}
\toprule
Cluster & Error Medio (RMSE) & Error Absoluto (MAE) & Variabilidad Explicada (R²)\\
\midrule
1 & 31.48091 & 24.61647 & 0.2850278\\
2 & 33.13467 & 23.68081 & 0.1734142\\
3 & 36.64418 & 27.36977 & 0.1900675\\
4 & 29.89712 & 22.78005 & 0.2483261\\
\bottomrule
\end{longtable}

\begin{Shaded}
\begin{Highlighting}[]
\CommentTok{\# Suponiendo que test ya tiene: precio\_real, precio\_predicho y cluster}
\NormalTok{resultados\_cluster }\OtherTok{\textless{}{-}}\NormalTok{ test }\SpecialCharTok{\%\textgreater{}\%}
  \FunctionTok{mutate}\NormalTok{(}
    \AttributeTok{precio\_real =} \FunctionTok{exp}\NormalTok{(log\_precio) }\SpecialCharTok{{-}} \DecValTok{1}\NormalTok{,}
    \AttributeTok{precio\_predicho =}\NormalTok{ pred}
\NormalTok{  )}

\CommentTok{\# Gráfico con facetas por cluster}
\FunctionTok{ggplot}\NormalTok{(resultados\_cluster, }\FunctionTok{aes}\NormalTok{(}\AttributeTok{x =}\NormalTok{ precio\_real, }\AttributeTok{y =}\NormalTok{ precio\_predicho)) }\SpecialCharTok{+}
  \FunctionTok{geom\_point}\NormalTok{(}\AttributeTok{alpha =} \FloatTok{0.4}\NormalTok{, }\AttributeTok{color =} \StringTok{"\#1f78b4"}\NormalTok{, }\AttributeTok{size =} \FloatTok{1.5}\NormalTok{) }\SpecialCharTok{+}
  \FunctionTok{geom\_abline}\NormalTok{(}\AttributeTok{slope =} \DecValTok{1}\NormalTok{, }\AttributeTok{intercept =} \DecValTok{0}\NormalTok{, }\AttributeTok{linetype =} \StringTok{"dashed"}\NormalTok{, }\AttributeTok{color =} \StringTok{"red"}\NormalTok{) }\SpecialCharTok{+}
  \FunctionTok{facet\_wrap}\NormalTok{(}\SpecialCharTok{\textasciitilde{}}\NormalTok{cluster) }\SpecialCharTok{+}
  \FunctionTok{labs}\NormalTok{(}
    \AttributeTok{title =} \StringTok{"Precio Real vs Precio Predicho por Cluster"}\NormalTok{,}
    \AttributeTok{x =} \StringTok{"Precio Real (€)"}\NormalTok{,}
    \AttributeTok{y =} \StringTok{"Precio Predicho (€)"}
\NormalTok{  ) }\SpecialCharTok{+}
  \FunctionTok{theme\_minimal}\NormalTok{(}\AttributeTok{base\_size =} \DecValTok{14}\NormalTok{)}
\end{Highlighting}
\end{Shaded}

\includegraphics{Predicción_precio_vivienda_files/figure-latex/Real v Predicho por clúster-1.pdf}

\begin{Shaded}
\begin{Highlighting}[]
\NormalTok{resultados\_barrios }\OtherTok{\textless{}{-}}\NormalTok{ datos\_modelo }\SpecialCharTok{\%\textgreater{}\%}
  \FunctionTok{split}\NormalTok{(.}\SpecialCharTok{$}\NormalTok{barrio) }\SpecialCharTok{\%\textgreater{}\%}
  \FunctionTok{map\_df}\NormalTok{(}\ControlFlowTok{function}\NormalTok{(df) \{}
    \ControlFlowTok{if}\NormalTok{ (}\FunctionTok{nrow}\NormalTok{(df) }\SpecialCharTok{\textless{}} \DecValTok{20}\NormalTok{) }\FunctionTok{return}\NormalTok{(}\ConstantTok{NULL}\NormalTok{)  }\CommentTok{\# evitar barrios con pocos datos}
    
    \FunctionTok{set.seed}\NormalTok{(}\DecValTok{123}\NormalTok{)}
\NormalTok{    train\_indices }\OtherTok{\textless{}{-}} \FunctionTok{sample}\NormalTok{(}\DecValTok{1}\SpecialCharTok{:}\FunctionTok{nrow}\NormalTok{(df), }\AttributeTok{size =} \FloatTok{0.8} \SpecialCharTok{*} \FunctionTok{nrow}\NormalTok{(df))}
\NormalTok{    train }\OtherTok{\textless{}{-}}\NormalTok{ df[train\_indices, ]}
\NormalTok{    test }\OtherTok{\textless{}{-}}\NormalTok{ df[}\SpecialCharTok{{-}}\NormalTok{train\_indices, ]}
    
\NormalTok{    x\_vars }\OtherTok{\textless{}{-}} \FunctionTok{c}\NormalTok{(}\StringTok{"bedrooms"}\NormalTok{, }\StringTok{"bathrooms"}\NormalTok{, }\StringTok{"accommodates"}\NormalTok{,}
                \StringTok{"precio\_ponderado\_barrio\_media"}\NormalTok{,}
                \StringTok{"fiabilidad\_normalizada"}\NormalTok{,}
                \StringTok{"puntuacion\_media\_normal\_reseñas"}\NormalTok{)}
    
\NormalTok{    X\_train }\OtherTok{\textless{}{-}} \FunctionTok{as.matrix}\NormalTok{(}\FunctionTok{sapply}\NormalTok{(train[, x\_vars], as.numeric))}
\NormalTok{    X\_test }\OtherTok{\textless{}{-}} \FunctionTok{as.matrix}\NormalTok{(}\FunctionTok{sapply}\NormalTok{(test[, x\_vars], as.numeric))}
\NormalTok{    y\_train }\OtherTok{\textless{}{-}} \FunctionTok{log}\NormalTok{(train}\SpecialCharTok{$}\NormalTok{price }\SpecialCharTok{+} \DecValTok{1}\NormalTok{)}
\NormalTok{    y\_test }\OtherTok{\textless{}{-}} \FunctionTok{log}\NormalTok{(test}\SpecialCharTok{$}\NormalTok{price }\SpecialCharTok{+} \DecValTok{1}\NormalTok{)}
    
\NormalTok{    modelo }\OtherTok{\textless{}{-}} \FunctionTok{xgboost}\NormalTok{(}
      \AttributeTok{data =}\NormalTok{ X\_train,}
      \AttributeTok{label =}\NormalTok{ y\_train,}
      \AttributeTok{nrounds =} \DecValTok{100}\NormalTok{,}
      \AttributeTok{max\_depth =} \DecValTok{4}\NormalTok{,}
      \AttributeTok{eta =} \FloatTok{0.1}\NormalTok{,}
      \AttributeTok{objective =} \StringTok{"reg:squarederror"}\NormalTok{,}
      \AttributeTok{verbose =} \DecValTok{0}
\NormalTok{    )}
    
\NormalTok{    pred }\OtherTok{\textless{}{-}} \FunctionTok{exp}\NormalTok{(}\FunctionTok{predict}\NormalTok{(modelo, X\_test)) }\SpecialCharTok{{-}} \DecValTok{1}
\NormalTok{    real }\OtherTok{\textless{}{-}} \FunctionTok{exp}\NormalTok{(y\_test) }\SpecialCharTok{{-}} \DecValTok{1}
    
\NormalTok{    resultado }\OtherTok{\textless{}{-}} \FunctionTok{tibble}\NormalTok{(}
      \AttributeTok{barrio =} \FunctionTok{unique}\NormalTok{(df}\SpecialCharTok{$}\NormalTok{barrio),}
      \AttributeTok{RMSE =} \FunctionTok{rmse}\NormalTok{(real, pred),}
      \AttributeTok{MAE =} \FunctionTok{mae}\NormalTok{(real, pred),}
      \AttributeTok{R2 =} \FunctionTok{cor}\NormalTok{(real, pred)}\SpecialCharTok{\^{}}\DecValTok{2}
\NormalTok{    )}
    
    \FunctionTok{print}\NormalTok{(resultado)  }\CommentTok{\# Mostrar resultado por consola}
    \FunctionTok{return}\NormalTok{(resultado)}
\NormalTok{  \})}
\end{Highlighting}
\end{Shaded}

\begin{verbatim}
## # A tibble: 1 x 4
##   barrio  RMSE   MAE    R2
##   <chr>  <dbl> <dbl> <dbl>
## 1 AIORA   30.4  18.6 0.393
## # A tibble: 1 x 4
##   barrio  RMSE   MAE    R2
##   <chr>  <dbl> <dbl> <dbl>
## 1 ALBORS  36.5  25.2 0.119
## # A tibble: 1 x 4
##   barrio       RMSE   MAE    R2
##   <chr>       <dbl> <dbl> <dbl>
## 1 ARRANCAPINS  36.3  26.9 0.168
## # A tibble: 1 x 4
##   barrio     RMSE   MAE    R2
##   <chr>     <dbl> <dbl> <dbl>
## 1 BENICALAP  53.9  30.8 0.186
## # A tibble: 1 x 4
##   barrio      RMSE   MAE    R2
##   <chr>      <dbl> <dbl> <dbl>
## 1 BENIMACLET  31.6  28.7 0.240
## # A tibble: 1 x 4
##   barrio               RMSE   MAE    R2
##   <chr>               <dbl> <dbl> <dbl>
## 1 CABANYAL-CANYAMELAR  25.6  19.4 0.382
## # A tibble: 1 x 4
##   barrio                                RMSE   MAE    R2
##   <chr>                                <dbl> <dbl> <dbl>
## 1 CIUTAT DE LES ARTS I DE LES CIENCIES  17.3  13.1 0.725
## # A tibble: 1 x 4
##   barrio        RMSE   MAE     R2
##   <chr>        <dbl> <dbl>  <dbl>
## 1 CIUTAT JARDI  19.1  8.52 0.0262
## # A tibble: 1 x 4
##   barrio      RMSE   MAE    R2
##   <chr>      <dbl> <dbl> <dbl>
## 1 EL BOTANIC  29.3  20.6 0.214
## # A tibble: 1 x 4
##   barrio    RMSE   MAE    R2
##   <chr>    <dbl> <dbl> <dbl>
## 1 EL CARME  24.6  15.7 0.481
## # A tibble: 1 x 4
##   barrio   RMSE   MAE      R2
##   <chr>   <dbl> <dbl>   <dbl>
## 1 EL GRAU  23.8  17.0 0.00193
## # A tibble: 1 x 4
##   barrio     RMSE   MAE     R2
##   <chr>     <dbl> <dbl>  <dbl>
## 1 EL MERCAT  39.5  30.3 0.0902
## # A tibble: 1 x 4
##   barrio         RMSE   MAE       R2
##   <chr>         <dbl> <dbl>    <dbl>
## 1 EL PERELLONET  22.1  17.9 0.000403
## # A tibble: 1 x 4
##   barrio    RMSE   MAE    R2
##   <chr>    <dbl> <dbl> <dbl>
## 1 EL PILAR  38.7  30.2 0.106
## # A tibble: 1 x 4
##   barrio    RMSE   MAE    R2
##   <chr>    <dbl> <dbl> <dbl>
## 1 EL SALER  58.6  52.5 0.843
## # A tibble: 1 x 4
##   barrio    RMSE   MAE    R2
##   <chr>    <dbl> <dbl> <dbl>
## 1 EN CORTS  22.5  16.5 0.527
## # A tibble: 1 x 4
##   barrio            RMSE   MAE    R2
##   <chr>            <dbl> <dbl> <dbl>
## 1 LA CREU DEL GRAU  41.0  28.3 0.125
## # A tibble: 1 x 4
##   barrio         RMSE   MAE     R2
##   <chr>         <dbl> <dbl>  <dbl>
## 1 LA MALVA-ROSA  51.1  39.2 0.0234
## # A tibble: 1 x 4
##   barrio      RMSE   MAE    R2
##   <chr>      <dbl> <dbl> <dbl>
## 1 LA PETXINA  12.5  10.1 0.722
## # A tibble: 1 x 4
##   barrio     RMSE   MAE     R2
##   <chr>     <dbl> <dbl>  <dbl>
## 1 LA RAIOSA  39.5  26.3 0.0467
## # A tibble: 1 x 4
##   barrio      RMSE   MAE     R2
##   <chr>      <dbl> <dbl>  <dbl>
## 1 LA ROQUETA  33.9  29.2 0.0226
## # A tibble: 1 x 4
##   barrio  RMSE   MAE       R2
##   <chr>  <dbl> <dbl>    <dbl>
## 1 LA SEU  42.1  32.6 0.000156
## # A tibble: 1 x 4
##   barrio    RMSE   MAE     R2
##   <chr>    <dbl> <dbl>  <dbl>
## 1 LA XEREA  51.8  39.8 0.0496
## # A tibble: 1 x 4
##   barrio    RMSE   MAE     R2
##   <chr>    <dbl> <dbl>  <dbl>
## 1 MESTALLA  47.3  35.9 0.0226
## # A tibble: 1 x 4
##   barrio       RMSE   MAE    R2
##   <chr>       <dbl> <dbl> <dbl>
## 1 MONT-OLIVET  35.5  24.6 0.165
## # A tibble: 1 x 4
##   barrio    RMSE   MAE    R2
##   <chr>    <dbl> <dbl> <dbl>
## 1 MORVEDRE  35.4  28.5 0.109
## # A tibble: 1 x 4
##   barrio    RMSE   MAE    R2
##   <chr>    <dbl> <dbl> <dbl>
## 1 NATZARET  24.1  19.8 0.235
## # A tibble: 1 x 4
##   barrio     RMSE   MAE     R2
##   <chr>     <dbl> <dbl>  <dbl>
## 1 NOU MOLES  56.7  42.1 0.0370
## # A tibble: 1 x 4
##   barrio   RMSE   MAE    R2
##   <chr>   <dbl> <dbl> <dbl>
## 1 PATRAIX  35.3  28.3 0.268
## # A tibble: 1 x 4
##   barrio      RMSE   MAE     R2
##   <chr>      <dbl> <dbl>  <dbl>
## 1 PENYA-ROJA  46.0  34.2 0.0114
## # A tibble: 1 x 4
##   barrio   RMSE   MAE    R2
##   <chr>   <dbl> <dbl> <dbl>
## 1 RUSSAFA  27.3  18.6 0.368
## # A tibble: 1 x 4
##   barrio         RMSE   MAE    R2
##   <chr>         <dbl> <dbl> <dbl>
## 1 SANT FRANCESC  29.2  25.2 0.505
## # A tibble: 1 x 4
##   barrio    RMSE   MAE      R2
##   <chr>    <dbl> <dbl>   <dbl>
## 1 TRINITAT  60.8  40.5 0.00434
\end{verbatim}

\begin{Shaded}
\begin{Highlighting}[]
\CommentTok{\# Tabla final}
\NormalTok{resultados\_barrios }\SpecialCharTok{\%\textgreater{}\%}
  \FunctionTok{rename}\NormalTok{(}\StringTok{"Barrios"} \OtherTok{=}\NormalTok{ barrio, }\StringTok{"Error Medio (RMSE)"} \OtherTok{=}\NormalTok{ RMSE,}
         \StringTok{"Error Absoluto (MAE)"} \OtherTok{=}\NormalTok{ MAE, }\StringTok{"Variabilidad Explicada (R²)"} \OtherTok{=}\NormalTok{ R2) }\SpecialCharTok{\%\textgreater{}\%}
  \FunctionTok{kable}\NormalTok{() }\SpecialCharTok{\%\textgreater{}\%}
  \FunctionTok{kable\_styling}\NormalTok{(}\AttributeTok{full\_width =} \ConstantTok{FALSE}\NormalTok{)}
\end{Highlighting}
\end{Shaded}

\begin{longtable}[t]{lrrr}
\toprule
Barrios & Error Medio (RMSE) & Error Absoluto (MAE) & Variabilidad Explicada (R²)\\
\midrule
AIORA & 30.44133 & 18.644351 & 0.3931851\\
ALBORS & 36.51812 & 25.194352 & 0.1187561\\
ARRANCAPINS & 36.31233 & 26.900144 & 0.1679798\\
BENICALAP & 53.94661 & 30.772631 & 0.1855498\\
BENIMACLET & 31.61886 & 28.690716 & 0.2395053\\
\addlinespace
CABANYAL-CANYAMELAR & 25.57267 & 19.360126 & 0.3816936\\
CIUTAT DE LES ARTS I DE LES CIENCIES & 17.29255 & 13.097678 & 0.7245705\\
CIUTAT JARDI & 19.12837 & 8.515997 & 0.0261818\\
EL BOTANIC & 29.29469 & 20.619187 & 0.2139596\\
EL CARME & 24.55420 & 15.734508 & 0.4807880\\
\addlinespace
EL GRAU & 23.79839 & 16.966716 & 0.0019255\\
EL MERCAT & 39.50692 & 30.290885 & 0.0901749\\
EL PERELLONET & 22.12627 & 17.899643 & 0.0004027\\
EL PILAR & 38.69979 & 30.189332 & 0.1061949\\
EL SALER & 58.60096 & 52.537784 & 0.8433180\\
\addlinespace
EN CORTS & 22.45521 & 16.514202 & 0.5266626\\
LA CREU DEL GRAU & 40.96666 & 28.334574 & 0.1245105\\
LA MALVA-ROSA & 51.10258 & 39.186370 & 0.0233792\\
LA PETXINA & 12.47599 & 10.086840 & 0.7223119\\
LA RAIOSA & 39.53469 & 26.276769 & 0.0467033\\
\addlinespace
LA ROQUETA & 33.87655 & 29.197555 & 0.0226299\\
LA SEU & 42.10095 & 32.611037 & 0.0001562\\
LA XEREA & 51.81800 & 39.784649 & 0.0496458\\
MESTALLA & 47.30057 & 35.879529 & 0.0226024\\
MONT-OLIVET & 35.54812 & 24.584436 & 0.1649057\\
\addlinespace
MORVEDRE & 35.39420 & 28.480890 & 0.1087295\\
NATZARET & 24.13775 & 19.840031 & 0.2349901\\
NOU MOLES & 56.74781 & 42.097554 & 0.0369785\\
PATRAIX & 35.34333 & 28.335936 & 0.2677295\\
PENYA-ROJA & 46.04178 & 34.178901 & 0.0113643\\
\addlinespace
RUSSAFA & 27.29586 & 18.612827 & 0.3675977\\
SANT FRANCESC & 29.22808 & 25.208461 & 0.5054967\\
TRINITAT & 60.82024 & 40.516916 & 0.0043411\\
\bottomrule
\end{longtable}

\begin{Shaded}
\begin{Highlighting}[]
\CommentTok{\# Calcular error}
\NormalTok{resultados }\OtherTok{\textless{}{-}}\NormalTok{ test }\SpecialCharTok{\%\textgreater{}\%}
  \FunctionTok{mutate}\NormalTok{(}
    \AttributeTok{precio\_real =} \FunctionTok{exp}\NormalTok{(log\_precio) }\SpecialCharTok{{-}} \DecValTok{1}\NormalTok{,}
    \AttributeTok{precio\_predicho =}\NormalTok{ pred,}
    \AttributeTok{error =}\NormalTok{ precio\_predicho }\SpecialCharTok{{-}}\NormalTok{ precio\_real}
\NormalTok{  )}

\CommentTok{\# Histograma}
\FunctionTok{ggplot}\NormalTok{(resultados, }\FunctionTok{aes}\NormalTok{(}\AttributeTok{x =}\NormalTok{ error)) }\SpecialCharTok{+}
  \FunctionTok{geom\_histogram}\NormalTok{(}\AttributeTok{fill =} \StringTok{"\#1f78b4"}\NormalTok{, }\AttributeTok{bins =} \DecValTok{30}\NormalTok{, }\AttributeTok{alpha =} \FloatTok{0.7}\NormalTok{) }\SpecialCharTok{+}
  \FunctionTok{geom\_vline}\NormalTok{(}\AttributeTok{xintercept =} \DecValTok{0}\NormalTok{, }\AttributeTok{color =} \StringTok{"red"}\NormalTok{, }\AttributeTok{linetype =} \StringTok{"dashed"}\NormalTok{) }\SpecialCharTok{+}
  \FunctionTok{labs}\NormalTok{(}\AttributeTok{title =} \StringTok{"Distribución del error (Precio Predicho − Real)"}\NormalTok{,}
       \AttributeTok{x =} \StringTok{"Error (€)"}\NormalTok{, }\AttributeTok{y =} \StringTok{"Frecuencia"}\NormalTok{) }\SpecialCharTok{+}
  \FunctionTok{theme\_minimal}\NormalTok{()}
\end{Highlighting}
\end{Shaded}

\includegraphics{Predicción_precio_vivienda_files/figure-latex/Distribución del error-1.pdf}
El histograma muestra una distribución asimétrica a la izquierda, con la
mayoría de los errores concentrados entre -50€ y +50€.

\begin{Shaded}
\begin{Highlighting}[]
\NormalTok{mean\_error }\OtherTok{\textless{}{-}} \FunctionTok{mean}\NormalTok{(resultados}\SpecialCharTok{$}\NormalTok{error)}
\FunctionTok{cat}\NormalTok{(}\StringTok{"Error medio:"}\NormalTok{, }\FunctionTok{round}\NormalTok{(mean\_error, }\DecValTok{2}\NormalTok{), }\StringTok{"€"}\NormalTok{)}
\end{Highlighting}
\end{Shaded}

\begin{verbatim}
## Error medio: -5.93 €
\end{verbatim}

El error medio es de -6.1 €, lo que indica una ligera tendencia a
subestimar el precio real en promedio.

\begin{Shaded}
\begin{Highlighting}[]
\CommentTok{\# Scatterplot de error absoluto vs precio real}
\FunctionTok{ggplot}\NormalTok{(resultados, }\FunctionTok{aes}\NormalTok{(}\AttributeTok{x =}\NormalTok{ precio\_real, }\AttributeTok{y =} \FunctionTok{abs}\NormalTok{(error))) }\SpecialCharTok{+}
  \FunctionTok{geom\_point}\NormalTok{(}\AttributeTok{alpha =} \FloatTok{0.3}\NormalTok{, }\AttributeTok{color =} \StringTok{"\#fc8d62"}\NormalTok{) }\SpecialCharTok{+}
  \FunctionTok{geom\_smooth}\NormalTok{(}\AttributeTok{method =} \StringTok{"loess"}\NormalTok{, }\AttributeTok{se =} \ConstantTok{FALSE}\NormalTok{, }\AttributeTok{color =} \StringTok{"darkred"}\NormalTok{) }\SpecialCharTok{+}
  \FunctionTok{labs}\NormalTok{(}\AttributeTok{title =} \StringTok{"Error absoluto según el precio real"}\NormalTok{,}
       \AttributeTok{x =} \StringTok{"Precio real (€)"}\NormalTok{,}
       \AttributeTok{y =} \StringTok{"Error absoluto (€)"}\NormalTok{) }\SpecialCharTok{+}
  \FunctionTok{theme\_minimal}\NormalTok{()}
\end{Highlighting}
\end{Shaded}

\begin{verbatim}
## `geom_smooth()` using formula = 'y ~ x'
\end{verbatim}

\includegraphics{Predicción_precio_vivienda_files/figure-latex/Error absoluto según precio real-1.pdf}
El gráfico de dispersión muestra que el error absoluto tiende a aumentar
con el precio real:

En viviendas con precio real \textless100€, el modelo es bastante
preciso (errores bajos). En viviendas \textgreater150€, los errores se
amplifican notablemente.

El modelo predice bien en el rango medio-bajo de precios, pero su
rendimiento disminuye conforme el precio aumenta.

\end{document}
